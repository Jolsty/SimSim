\documentclass[11pt]{article}

\usepackage[a4paper]{geometry}
\usepackage[italian]{babel}
\usepackage{tesi}
\usepackage[spaces, hyphens, obeyspaces]{url}
\usepackage[pdfa]{hyperref}
\usepackage{colorprofiles}
\usepackage[a-1a]{pdfx}

\RequirePackage{filecontents}
\begin{filecontents*}{\jobname.xmpdata}
	\Title{Riassunto di tesi: Sistema Accessi IoT}
	\Author{Andrei Ciulpan}
	\Language{it-IT}
	\Subject{Riassunto di Tesi di Laurea Triennale}
	\Keywords{Sviluppo Web\sep Sistemi Embedded\sep MongoDB\sep Arduino\sep Node.js}
\end{filecontents*}

\renewcommand{\UrlFont}{\footnotesize}


\begin{document}
	\noindent\textbf{\large 1. Ente presso cui è stato svolto il lavoro di stage} \\
	\newline 
	Il lavoro di stage è stato svolto presso \emph{Yatta}, il primo makerspace in centro a Milano. Yatta è uno spazio attrezzato con tecnologie e macchinari digitali destinato alla prototipazione
	e gestito da professionisti del settore 3D, programmazione, elettronica e grafica, in una
	visione ampliata del concetto di artigiano orientato verso la Manifattura e Industria
	4.0.  \\
	\newline
	\noindent\textbf{\large 2. Contesto iniziale} \\
	\newline 
	Sono stato inserito all'interno dell'azienda per lavorare su una versione iniziale già esistente ma incompleta e, secondo me, abbastanza incasinata a livello di codice scritto, perciò si è deciso di iniziare il tutto da capo utilizzando quasi tutti gli stessi componenti.  \\
	\newline
	\noindent\textbf{\large 3. Obiettivi del lavoro} \\
	\newline 
	L’attività principale del tirocinio è stata la realizzazione di un sistema di controllo accessi basato su Arduino o MCU analogo, da applicare presso varchi. I requisiti principali sono stati: 
	\begin{itemize}
		\item diverse modalità di riconoscimento: tessere RFID, tastierino numerico, telecomando;
		\item a seguito di ogni accesso avvenuto con successo, deve aprire la serratura di una porta;
		\item a seguito di ogni accesso avvenuto con successo, deve salvare il log su scheda SD o su sistema remoto (LAN o Cloud) connesso via WiFi.  \\ 
	\end{itemize} 

	\noindent\textbf{\large 4. Descrizione del lavoro svolto} \\
	\newline 
	Il frutto di questo progetto è stato il prototipo di un sistema di controllo accessi IoT \cite{IoT,smart_objects, controllo_accessi} per l'azienda ospitante del tirocinio. Durante lo sviluppo del progetto, l'idea è evoluta: oltre ai requisiti iniziali, è stato sviluppato anche un server/database ospitato nella LAN di una rete privata accessibile solo dall'amministratore. Il database è stato creato per memorizzare i log, i soci e le tessere dell'azienda. Inoltre è stata sviluppata un'interfaccia grafica (in questo caso un sito web in locale) per gestire il database tramite semplici operazioni CRUD. \\
	\newline
	\noindent\textbf{\large 5. Tecnologie coinvolte} \\
	\newline 
	Sistema embedded: 
	\begin{itemize}
		\item componenti: Arduino UNO\cite{sistemi_embedded_atrent}, Raspberry Pi, Ricevitore RF, Keypad, RTC, Servomotore, Lettore RFID, ESP8266\cite{esp_ds}; 
		\item linguaggi di programmazione: Arduino Programming Language.
	\end{itemize} 
	\pagebreak
	\noindent Sito web:
	\begin{itemize}
		\item design patterns: MVC, REST;
		\item tecnologie: Node.js e alcune sue librerie, MongoDB\cite{mongodb_driver, mongoose}, jQuery, W3CSS, AJAX;
		\item linguaggi di programmazione: HTML5, CSS, JavaScript (client-side e server-side).
		\\
	\end{itemize}
	\noindent\textbf{\large 6. Competenze e risultati raggiunti} \\
	\newline 
	\noindent Il prototipo è funzionante e abbastanza completo, ma prima di essere messo in funzione potrebbero essere necessari
	alcuni ritocchi e miglioramenti per renderlo più moderno secondo gli standard dei sistemi commerciali attualmente in uso.
	\newline
	L'esperienza di tirocinio è stata, secondo me, un successo e mi ha insegnato come lavorare in autonomia ma allo stesso tempo tenendo conto anche del teamworking (discussione degli specifici e delle varie tecnologie da usare, creazione del modello 3D del prototipo assieme alla collega).
	\newline
	
	\noindent Problemi affrontati:
	\begin{itemize}
		\item incompatibilità tra Raspbian (sistema operativo su 32 bit) e le ultime versioni di MongoDB $\rightarrow$ installazione del sistema operativo openSUSE (64 bit) sulla Raspberry Pi
		\item problema di sicurezza all'interno della rete locale dovuta alla mancanza di un sistema di login $\rightarrow$ rete LAN privata accessibile solo dall'amministratore
		\item disattivazione automatica delle tessere dei soci non più iscritti $\rightarrow$ implementazione di uno scheduler
		\item problema dei timestamp nei log dovuto al fatto che la Raspberry Pi senza connessione Internet non è in grado di tenere traccia del tempo $\rightarrow$ RTC
	\end{itemize}

\begin{thebibliography}{00}
	%
	\bibitem{IoT}
	M. Rouse, Internet of Things (IoT), ultimo aggiornamento Marzo 2019.
	\url{https://internetofthingsagenda.techtarget.com/definition/Internet-of-Things-IoT}
	% 
	
	%
	\bibitem{smart_objects}
	A. Tumino, Internet of Things: gli oggetti intelligenti prima di ogni \textquotedblleft{} cosa \textquotedblright{}, 24 Gennaio, 2018.
	\url{https://blog.osservatori.net/it\_it/internet-of-things-oggetti-intelligenti-prima-di-ogni-cosa}
	%

	%
	\bibitem{controllo_accessi}
	J. Allen, Opening new doors with IP access control, 16 Marzo, 2018. \url{https://www.axis.com/blog/secure-insights/opening-new-doors-with-ip-access-control/}
	%
	
	%
	\bibitem{sistemi_embedded_atrent}
	A. Carraturo, A. Trentini, Sistemi Embedded: Teoria e Pratica, prima edizione: Settembre 2017.
	\url{http://www.ledizioni.it/prodotto/a-carraturo-a-trentini-sistemi-embedded-teoria-pratica/}
	%
	
	%
	\bibitem{esp_ds}
	Ai-Thinker Team, ESP-01 WiFi Module (Version1.0).
	\url{http://www.microchip.ua/wireless/esp01.pdf}
	%
	
	%
	\bibitem{mongodb_driver}
	MongoDB, MongoDB Node.js Driver Documentation v3.2
	(\url{https://mongodb.github.io/node-mongodb-native/3.2/}
	%
	
	%
	\bibitem{mongoose}
	Mongoose, elegant mongodb object modeling for node.js.
	\url{https://mongoosejs.com/}
	%
	
\end{thebibliography}
	
\end{document}