\documentclass[11pt]{report}

\usepackage[a4paper]{geometry}
\usepackage[italian]{babel}


\begin{document}
	
	\noindent\textbf{\large RIASSUNTO TESI DI LAUREA \textquotedblleft{} SISTEMA ACCESSI IOT \textquotedblright{}} 
	\vspace*{30px} \\
	\textbf{Corso}: Informatica F1X - Laurea Triennale \\
	\textbf{Laureando}: Andrei CIULPAN \\
	\textbf{Matricola}: 872394 \\
	\textbf{Relatore}: Dott. Andrea TRENTINI \\
	\textbf{Correlatore}: Marco LANZA \\
	\textbf{Data di termine dello stage}: 22 Febbraio 2019
	\vspace*{30px} \\
	
	\noindent\textbf{\large 1. Ente presso cui è stato svolto il lavoro di stage} \\
	\newline 
	Il lavoro di stage è stato svolto presso \emph{Yatta}, il primo makerspace in centro a Milano. Yatta è uno spazio attrezzato con tecnologie e macchinari digitali destinato alla prototipazione
	e gestito da professionisti del settore 3D, programmazione, elettronica e grafica, in una
	visione ampliata del concetto di artigiano orientato verso la Manifattura e Industria
	4.0.  \\
	\newline
	\noindent\textbf{\large 2. Contesto iniziale} \\
	\newline 
	Sono stato inserito all'interno dell'azienda per lavorare su una versione iniziale già esistente ma incompleta e, secondo me, abbastanza incasinata a livello di codice scritto, perciò si è deciso di iniziare il tutto da capo utilizzando quasi tutti gli stessi componenti.  \\
	\newline
	\noindent\textbf{\large 3. Obiettivi del lavoro} \\
	\newline 
	L’attività principale del tirocinio è stata la realizzazione di un sistema di controllo accessi IoT,
	da applicare presso varchi. I requisiti principali sono stati: 
	\begin{itemize}
		\item diverse modalità di riconoscimento: tessere RFID, tastierino numerico, telecomando;
		\item database per salvare i log/soci/tessere (in questo caso il database risiede in locale);
		\item interfaccia grafica (in questo caso sito web) per gestire il database tramite operazioni CRUD;
		\item a seguito di ogni accesso avvenuto con successo, deve aprire la serratura di una porta e salvare il log nel database.
	\end{itemize} 
	\pagebreak
	\noindent\textbf{\large 4. Descrizione del lavoro svolto} \\
	\newline 
	a \\
	\newline
	\noindent\textbf{\large 5. Tecnologie coinvolte} \\
	\newline 
	a \\
	\newline
	\noindent\textbf{\large 6. Competenze e risultati raggiunti} \\
	\newline 
	a \\
	\newline
	\noindent\textbf{\large 7. Bibliografia} \\
	\newline 
	Alcuni riferimenti utilizzati: \\ 
	\newline
	
	
	
%	\noindent L'obiettivo della tesi è quello di analizzare il prototipo di un sistema di controllo accessi creato durante l'attività di tirocinio presso \emph{Yatta}. Si tratta di un sistema \emph{Internet of Things} (IoT), ovvero un sistema in cui la connessione Internet viene estesa anche al mondo degli oggetti (e.g. porte, macchine ecc.). I cosiddetti oggetti \textquotedblleft{} intelligenti \textquotedblright{} %
	
\end{document}